\documentclass[12pt,a4paper]{report}
\usepackage[utf8]{inputenc}
\usepackage{graphicx}



\begin{document} 


\begin{center}
    \Large{\sffamily{\textbf{NATIONAL INSTITUTE OF TECHNOLOGY}}}\\
\end{center}

\begin{center}
    \Large{\sffamily{RAIPUR , CHHATTISGARH}}\\ 
\end{center}

\begin{figure}
    \centering
    \includegraphics[scale=0.5]{nitlogo.png}
\end{figure}

\begin{center}
   \huge{\texttt{Assignment-3\\ on\\ "THE FUTURE OF HEALTHCARE"}}
  \end{center}
 
  
\begin{center}
\textbf{\underline{SUBMITTED BY:-}}\\

NAME:- MILOY KUMAR MANDAL\\
ROLL NO:- 21111032\\
BRANCH:- BIOMEDICAL ENGINEERING\\
SECTION:- A\\
SEMESTER:- 1ST\\
YEAR:- 1ST\\

\textbf{\underline{SUBMITTED TO:-}}\\
MR.SAURABH GUPTA\\
BIOMEDICAL DEPARTMENT\\
NIT RAIPUR\\


 
\end{center} 
\clearpage

\begin{center}
  \huge{\textbf{FUTURE OF HEALTHCARE}}
\end{center}



Health care is of course a topic of great interest and concern globally, as the confluence of macro factors have converged to make it one of the most significant issues facing leaders and citizens in all nations.The future of health will likely be driven by digital transformation enabled by radically interoperable data and open, secure platforms. Health is likely to revolve around sustaining well-being rather than responding to illness.\par 

TWENTY years from now, cancer and diabetes could join polio as defeated diseases. We expect prevention and early diagnoses will be central to the future of health. The onset of disease, in some cases, could be delayed or eliminated altogether. Sophisticated tests and tools could mean most diagnoses (and care) take place at home.\par



Not only will technology help cure illnesses, but the innovations will also boost the morale of people living with illnesses. This where MIRA steps in: a software that makes recovery and recuperation easier for patients, helping those who are coming through surgery or injury to regain their mobility as well as cognitive functions. With the introduction of gamification and virtual reality, patients will have fun and interactive ways to recover in the comfort of their home, remotely monitored by their therapists. This technology also tracks the process of their recovery,  providing valuable information that can be used to adjust and personalise therapy programmes for a speedier recovery.\par



The future of health that we envision is only about 20 years off, but health in 2040 will be a world apart from what we have now. Based on emerging technology, we can be reasonably certain that digital transformation—enabled by radically interoperable data, artificial intelligence (AI), and open, secure platforms—will drive much of this change. Unlike today, we believe care will be organized around the consumer, rather than around the institutions that drive our existing health care system.\par 

By 2040 (and perhaps beginning significantly before), streams of health data—together with data from a variety of other relevant sources—will merge to create a multifaceted and highly personalized picture of every consumer’s well-being. Today, wearable devices that track our steps, sleep patterns, and even heart rate have been integrated into our lives in ways we couldn’t have imagined just a few years ago. We expect this trend to accelerate. The next generation of sensors, for example, will move us from wearable devices to invisible, always-on sensors that are embedded in the devices that surround us.\par 

\begin{figure}
    \centering
    \includegraphics[scale=0.5]{download.jpg}
\end{figure}

\begin{figure}
    \centering
    \includegraphics[scale=0.5]{download1.jpg}
\end{figure}

Many medtech companies are already beginning to incorporate always-on biosensors and software into devices that can generate, gather, and share data. Advanced cognitive technologies could be developed to analyze a significantly large set of parameters and create personalized insights into a consumer’s health. The availability of data and personalized AI can enable precision well-being and real-time microinterventions that allow us to get ahead of sickness and far ahead of catastrophic disease.\par 

Consumers—armed with this highly detailed personal information about their own health—will likely demand that their health information be portable. Consumers have grown accustomed to transformations that have occurred in other sectors, such as e-commerce and mobility. These consumers will demand that health follow the same path and become an integrated part of their lives—and they’ll vote with their feet.\par 

The bubonic plague is a good example of a disease that can drastically change the healthcare system by quickly shifting all resources to handle an epidemic. In the Middle Ages, the Black Death spread so quickly across Europe that it is responsible for an estimated 75 million deaths. It may be surprising that the bubonic plague still circulates today. In fact, according to Center for Disease Control data, there were 11 cases and three deaths in the U.S. within five months in 2015.\par

So our healthcare system is developing very fast , and at this pace our healthcare is likely to face many innovations.


\end{document}
